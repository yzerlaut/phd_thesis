\documentclass[11pt]{book}
\usepackage[latin1, utf8]{inputenc}
\usepackage[french]{babel}
\setlength{\textwidth}{15cm}
\setlength{\textheight}{23cm}
% \setlength{\oddsidemargin}{-0cm}
% \setlength{\topmargin}{-1cm}

\title{}     
\date{} 
\author{}

\begin{document}


\thispagestyle{empty}

\vspace*{5cm}

\textit{``Pendant un certain temps, j'ai examin\'e les diff\'erentes
occupations auxquelles les hommes s'adonnent dans ce monde, et j'ai
essay\'e de choisir la meilleure. Mais il est inutile de raconter ici
quelles sont les pens\'ees qui me vinrent alors : qu'il me suffise de
dire que, pour ma part, rien ne me parut meilleur que
l'accomplissement rigoureux de mon dessein, \`a savoir : employer tout
le temps de ma vie \`a d\'evelopper ma raison et \`a rechercher les
traces de la v\'erit\'e ainsi que je me l'\'etais propos\'e. Car les
fruits que j'ai d\'ej\`a go\^ut\'es dans cette voie \'etaient tels
qu'\`a mon jugement, dans cette vie, rien ne peut \^etre trouv\'e de
plus agr\'eable et de plus innocent ; depuis que je me suis aid\'e de
cette sorte de m\'editation, chaque jour me fit d\'ecouvrir quelque
chose de nouveau qui avait quelque importance et n'\'etait point
g\'en\'eralement connu. C'est alors que mon \^ame devint si pleine de
joie que nulle autre chose ne pouvait lui importer."}

\vspace*{1cm}

\hspace{4cm} ---- Ren\'e Descartes, \textit{Le discours de la M\'ethode}

\newpage

% % now empty page
\thispagestyle{empty}
\quad 
\newpage

% % now dedicace page
% \thispagestyle{empty}
% \vspace*{9cm}
% \hspace{6cm} \textit{A mes parents,}
% \newpage

\thispagestyle{empty}

\huge \textbf{Remerciements} \normalsize
\vspace*{.5cm}

Je voudrais tout d'abord remercier mon directeur de thèse, Alain
Destexhe. Il y a 5 ans il m'a chaleuresement ouvert les portes de son
équipe et m'a initié avec enthousiasme au domaine des neurosciences
computationnelles. Bénéficier de son expertise et avoir son soutien
tout au long de cette thèse ont fait de ces années une expérience très
stimulante et enrichissante.

Ensuite je voudrais remercier Gilles Ouanounou. Il a pris le temps de
m'enseigner l'électrophysiologie intracellulaire et travailler à ses
côtés fut une très belle expérience scientifique et humaine. Je fais
le voeu que, grâce à son exemple, une part de son talent pour
l'observation et de sa créativité expérimentale ait réussit à
déteindre sur moi.

Bartosz Teleńczuk fut d'une grande aide tout au long de cette thèse:
de la \textit{charpenterie logiciel} à l'écriture scientifique, j'ai
beaucoup appris grâce à lui. Sa pertinence et sa sagesse ont été de
précieux atouts pour façonner une grande partie du contenu de cette
thèse.

Je remercie également Charlotte Deleuze et Thierry Bal qui, grâce à
leur disponibilité et leur enthousiasme, ont initié la partie
expérimentale de ma thèse. De manière plus ou moins directe, beaucoup
de membres du laboratoire on contribué à ce travail, pour n'en citer
que quelques uns, merci à Gérard, Guillaume, Kirsty, Aurélie, Manon,
Claude, Lyle, Marco, Michelle, Francesca, Sarah, ... Et bien sûr à
tous ceux que j'oublis.

Je souhaite remercier le laboratoire, en particulier ses directeurs
Yves Frégnac et Daniel Schulz, ils ont créé un endroit
particulièrement épanouissant pour l'étudiant curieux que je
suis. Venir au laboratoire fut un plaisir et je souhaite le meilleur à
tout le monde pour cette nouvelle aventure sur le plateau.

Je voudrais également remercier les collaborateurs extérieurs avec qui
nous avons travaillé durant cette thèse, intéragir avec eux fut une
réelle source de motivation: Maria-Victoria Sanchez-Vives au début de
ma thèse, puis Frédéric Chavane et Sandrine Chemla dans la seconde
moitié. Nos échanges furent très stimulant et je pense en avoir retiré
beaucoup.

Je remercie Ad Aersten et Jean-Marc Goaillard d'avoir accepté d'être
rapporteur de cette thèse, ainsi que Stéphane Charpier et Frédéric
Chavane d'avoir accepté d'éxaminer cette thèse. C'est un grand honneur
pour moi qu'ils participent à ce jury.

Cette thèse a été financé successivement par une bourse de
l'Initiative d'Excellence Paris-Saclay, puis par une bourse de la
Fondation pour la Recherche Médicale, je tiens à leur exprimé ma
gratitude.

Pour finir, je ne saurai assez remercier mes parents. Cette thèse est
le fruit des opportunités qu'ils m'ont offerts tout au long de ces
années, les mots me manquent pour les en remercier. Je remercie
également le reste de ma famille pour leur soutien bienveillant et
leurs encouragements.


\end{document}
